\documentclass[11pt]{mhletter}

\newcommand{\ui}[1]{\emph{\textbf{#1}}}

\begin{document}
\thispagestyle{empty}
% -*- TeX-master: "main"; fill-column: 72 -*-

\vspace*{-4.5em}
\begin{center}
  \textsc{\LARGE\textls[30]{California Institute of Technology}}\\
  {\footnotesize\textls[80]{DEPARTMENT OF COMPUTING AND MATHEMATICAL SCIENCES}}
\end{center}


\vspace*{4em}

\hspace{4.1in}\today

\vspace*{1em}

\begin{flushleft}
Thomas Lemberger\\
Molecular Systems Biology\\
EMBO Press
\end{flushleft}

\vspace*{1em}

\setlength{\parskip}{0.7em}

Dear Thomas, and reviewers,

We thank you for reviewing our manuscript on SBML Level~3 and providing many thoughtful comments and suggestions.  They helped us improve the manuscript.  Below you will find a point-by-point explanation of our responses to the reviewer's points; in addition, we have include a separate document showing the detailed differences (generated with the help of latexdiff) between the manuscript of July 2019 and the new version.  The differences include additional corrections and updates beyond the changes in response to the reviewer comments.

To the extent possible, we have endeavored to address all of the issues.  We hope that the reviewer's concerns have been addressed adequately and that the manuscript is now ready for publication.

% Ordering of signing authors is based on # of tasks assigned in the
% Google tracker document.
% MH: 6
% SK: 4
% DW: 4
% MK: 3
% CC: 2
% AD: 2
% FZ: 1
% CM: 1

\vspace*{1em}
\hspace{4in}
\begin{minipage}{3in}
With kind regards,\\
\\[1ex]
Michael Hucka\\
Dagmar Waltemath\\
Sarah Keating\\
Matthias K\"{o}nig\\
Claudine Chaouiya\\
Andreas Dr\"{a}ger\\
Fengkai Zhang\\
Chris Myers\\
on behalf of all the coauthors.
\end{minipage}

\clearpage

\begin{center}
\large Summary of changes
\end{center}

\textbf{\underline{Reviewer \#1}}

\textbf{\textit{1. When reading the title, the introduction of a newly developed format is expected. The abstract also gives the impression as if the manuscript would describe a recent development. However, on page 3 (lines 38-39) it is stated that SBML Level 3 was released in 2010 (and published in 2015?). To prevent such a misunderstanding, it would be better to rephrase title and abstract.}}

Thank you for pointing out the potential for misunderstanding.  In the revised version of the manuscript, we have edited the abstract to clarify that SBML Level~3 has already been in some use, and that this is not an entirely new development.  With respect to the title, we have tried to find an alternative, but could not find one that did not leave open the possibility of \emph{other} misinterpretations or misunderstandings.  We feel the title, while imperfect, is still the best we can do for a review article like this, and we hope the changes to the abstract are enough to avoid potential misunderstanding.


\textbf{\textit{2. The only new features of SBML Level 3 compared to previous levels that are discussed are the modularity comprising packages. Is this the only relevant feature that has been introduced?}}

We have updated the manuscript to emphasize that some other smaller changes have been made, though the modularity is the main change.  These points are made in a new paragraph at the end of the section titled ``The structure of SBML''.


\textbf{\textit{3. While the development of SBML is indeed a success story, the manuscript's tone tends to be overenthusiastic and to gloss over difficulties or problems. Are there alternative standards? How frequently refuse researchers to provide SBML files for their models? How is the support by journals (probably favorable, but this should be discussed)?}}

The goal of the manuscript is to describe the possibilities that SBML offers, but we did not intend to gloss over difficulties or problems.  The manuscript already mentioned some related standards, and we have now tried to clarify the relationships better in the edited second paragraph of the section titled ``Impact of SBML''.  Since a comparison to other formats is not the focus of the paper, and the journal's paper guidelines have length limitations, we hope this compromise is good enough.  With respect to journals, we have added text to the manuscript mentioning journal and funding agency guidelines.  We have also added more discussions of difficulties involving SBML.  These changes are in the second-to-last paragraph of the section titled ``Impact of SBML'' and the last three paragraphs of the section titled ``Forthcoming challenges''.


\textbf{\textit{Minor point: the individual parts of the manuscript, while well written, could be better connected.}}

To help connect the sections better, we have expanded the final paragraph in the introduction to provide a preview of the remaining sections, and in other parts of the text, we have added phrases referencing different sections in order to try to connect the sections better.


\clearpage
\textbf{\underline{Reviewer \#2}}

\textbf{\textit{1. When/why is standardization beneficial?}}


\textbf{\textit{2. Costs/benefits of the standardization process.}}


\textbf{\textit{3. The value of SBML as a technical platform vs.\ as a governing body.}}


\textbf{\textit{4. Human readability/usability of SBML.}}


\textbf{\textit{5. SBML as a knowledge representation vs. a model implementation.}}


\textbf{\textit{Minor comments.}}

We have updated the text to address the minor comments.

\end{document}
